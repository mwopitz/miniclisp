\label{implementierung}
\section[Implementierung]{Implementierung}

Wir haben uns bei der Entwicklung von Interpreter an der Referenzimplementierung
von Peter Norvig \ref{lispy} orientiert und eine ähnliche grundliegende
Struktur verwendet. Sowohl unser MiniCLisp-Interpreter als auch Norvigs
\emph{Lispy} verwenden eine einfache \emph{Read-eval-print-Schleife}, wobei
die wichtigste und interessanteste Funktion in beiden Implementierungen die
\texttt{eval}-Funktion ist, da in dieser die Eigentliche Scheme-Logik
umgesetzt wird.

\subsection[Datenstrukturen]{Verwendete Datenstrukturen}
Die zwei wichtigsten Datenstrukturen, die unser Interpreter verwendet, sind 
\emph{Expressions} (\texttt{struct expr}) und \emph{Environments} (\texttt{struct env}).
\par
Expressions werden verwendet um die typischen \emph{S-Expressions} von
Lisp und Scheme zu implementieren und sie können von der \texttt{eval}-Funktion
ausgewertet werden. Das \texttt{struct expr} sieht wie folgt aus:
\lstinputlisting[language=C, firstline=18, lastline=32]{../exploit/miniclisp.c}
Jede \texttt{expr} hat einen von mehreren Type der in dem Feld \texttt{type}
steht. Folgende Typen von Expressions verwendet unsere Implementierung:
\begin{lstlisting}[language=C]
enum exprtype {
  EXPRPROC, EXPRSYM, EXPRINT,
  EXPRLAMBDA, EXPRLIST, EXPREMPTY
};
\end{lstlisting}
Jenachdem welchen Type eine Expression hat, arbeiten wir jeweils nur mit einem
Feld aus der \texttt{union} innerhalb der \texttt{expr}:
\begin{itemize}
    \item Leere Expressions (d.h. \texttt{()}) haben den Typ \texttt{EXPREMPTY}
      und verwenden keines der Felder aus der \texttt{union}.
    \item Integerexpressions haben den Typ \texttt{EXPRINT} und verwenden den
      \texttt{intvalue}.
    \item Symbolic Expressions, d.h. Variablen, verwenden das
      \texttt{char}-Array \texttt{symvalue} für den Variablennamen.
    \item Listen von Expressions handhaben wir, indem wir eine \texttt{expr}
      mit dem Typ \texttt{EXPRLIST} anlegen und den \texttt{listptr} auf das
      erste Element in einer einfach verketteten Liste von \texttt{expr}s
      zeigen lassen. Für diese Liste setzen wir dann die \texttt{next}-Pointer
      korrekt und können sie so traversieren.
    \item Für Lambda-Expressions (\texttt{EXPRLAMBDA}) verwenden wir das
      \texttt{struct} innerhalb der \texttt{union} und setzen die Felder
      \texttt{lambdavars} auf eine Liste von \texttt{EXPRSYM}s (die Variablen
      für die Prozedur) und \texttt{lambdaexpr} auf eine \texttt{expr}, die
      den eigentlichen Körper der Prozedur darstellt.
    \item Für Expressions vom Typ \texttt{EXPRPROC} (\emph{Procedures}) setzen wir den
      Funktionspointer \texttt{proc}, womit wir im Prinzip beliebige
      C-Funktionen für den Interpreter zur Verfügung stellen können.
      Diese Art von Expressions spielt für unseren Exploit später eine
      wichtige Rolle.
      Der Benutzer von dem MiniCLisp-Interpreter kann auf diese Weise keine
      eigenen Prozeduren definieren, sondern muss für auf Lambdas
      zurückgreifen. Wir stellen jedoch mittels \texttt{EXPRPROC}s dem 
      Benutzer funktionen \texttt{+}, \texttt{*}, \texttt{<}, \texttt{gc}, usw.
      zur Verfügung.

\end{itemize}

