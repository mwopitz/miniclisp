\label{implementierung}
\section[Implementierung]{Implementierung}

Wir haben uns bei der Entwicklung von Interpreter an der Referenzimplementierung
von Peter Norvig \ref{lispy} orientiert und eine ähnliche grundliegende
Struktur verwendet. Sowohl unser MiniCLisp-Interpreter als auch Norvigs
\emph{Lispy} verwenden eine einfache \emph{Read-eval-print-Schleife}, wobei
die wichtigste und interessanteste Funktion in beiden Implementierungen die
\texttt{eval}-Funktion ist, da in dieser die Eigentliche Scheme-Logik
umgesetzt wird.

\subsection[Datenstrukturen]{Verwendete Datenstrukturen}
Die zwei wichtigsten Datenstrukturen, die unser Interpreter verwendet, sind 
\emph{Expressions} (\texttt{struct expr}) und \emph{Environments} (\texttt{struct env}).
\par
Expressions werden verwendet um die typischen \emph{S-Expressions} von
Lisp und Scheme zu implementieren und sie können von der \texttt{eval}-Funktion
ausgewertet werden. Das \texttt{struct expr} sieht wie folgt aus:
\lstinputlisting[language=C, firstline=20, lastline=35]{../exploit/miniclisp.c}
Jede \texttt{expr} hat einen von mehreren Type der in dem Feld \texttt{type}
steht. Folgende Typen von Expressions verwendet unsere Implementierung:
\begin{lstlisting}[language=C]
enum exprtype {
  EXPRPROC, EXPRSYM, EXPRINT,
  EXPRLAMBDA, EXPRLIST, EXPREMPTY
};
\end{lstlisting}
Jenachdem welchen Type eine Expression hat, arbeiten wir jeweils nur mit einem
Feld aus der \texttt{union} innerhalb der \texttt{expr}:
\begin{itemize}
    \item Leere Expressions (d.h. \texttt{()}) haben den Typ \texttt{EXPREMPTY}
      und verwenden keines der Felder aus der \texttt{union}.
    \item Integerexpressions haben den Typ \texttt{EXPRINT} und verwenden den
      \texttt{intvalue}.
    \item Symbolic Expressions, d.h. Variablen, verwenden das
      \texttt{char}-Array \texttt{symvalue} für den Variablennamen.
    \item Listen von Expressions handhaben wir, indem wir eine \texttt{expr}
      mit dem Typ \texttt{EXPRLIST} anlegen und den \texttt{listptr} auf das
      erste Element in einer einfach verketteten Liste von \texttt{expr}s
      zeigen lassen. Für diese Liste setzen wir dann die \texttt{next}-Pointer
      korrekt und können sie so traversieren.
    \item Für Lambda-Expressions (\texttt{EXPRLAMBDA}) verwenden wir das
      \texttt{struct} innerhalb der \texttt{union} und setzen die Felder
      \texttt{lambdavars} auf eine Liste von \texttt{EXPRSYM}s (die Variablen
      für die Prozedur) und \texttt{lambdaexpr} auf eine \texttt{expr}, die
      den eigentlichen Körper der Prozedur darstellt. Mit dem \texttt{env}-Pointer
      \texttt{lambdaenv} referenzieren wir das Environment, welches beim
      erstmaligen Auswerten einer Lambdaexpression angelegt wird, damit es für
      verschachtelte Lambdaexpressions zur Verfügung steht.
    \item Für Expressions vom Typ \texttt{EXPRPROC} (\emph{Procedures}) setzen wir den
      Funktionspointer \texttt{proc}, womit wir im Prinzip beliebige
      C-Funktionen für den Interpreter zur Verfügung stellen können.
      Diese Art von Expressions spielt für unseren Exploit später eine
      wichtige Rolle.
      Der Benutzer von dem MiniCLisp-Interpreter kann auf diese Weise keine
      eigenen Prozeduren definieren, sondern muss für auf Lambdas
      zurückgreifen. Wir stellen jedoch mittels \texttt{EXPRPROC}s dem 
      Benutzer funktionen \texttt{+}, \texttt{*}, \texttt{<}, \texttt{gc}, usw.
      zur Verfügung.
\end{itemize}
Das \texttt{in\_use}-Flag ist für die (nicht funktionierende) Garbage-Collection
notwendig.
\par
Die zweite wichtige Datenstruktur, die wir in unserem Interpreter verwenden,
sind Environments, die ein Mapping zwischen Variablennamen und deren Wert
darstellen. Diese werden insbesondere für \emph{Closures} benötigt, d.h. wenn
man bespielweise mehrer geklammerte Lambda-Expressions aufruft
\begin{lstlisting}[language=Scheme]
    (define f (lambda (a) (lambda (b) (+ a b))))
    ((f 1) 2)
\end{lstlisting}
Wir müssen, in diesem Fall, für die innere Lambda-Expression ein Environment
speichern, welches das Mapping \texttt{a -> 1} beinhaltet.
\par
Environments sind bei uns als simple \texttt{struct}s realisiert:
\lstinputlisting[language=C, firstline=43, lastline=47]{../exploit/miniclisp.c}
Jedes Environemnt enthält einen Pointer \texttt{outer} auf das umschließende
Environment (so dass wir bis hoch zum globalen Environment gehen können, wenn
ein Variablenname aufgelöst werden muss) und einen Pointer \texttt{list} der
auf ein simples Dictionary, in welchem die Mappings von Symbolnamen auf 
Symbolwerte gespeichert werden:
\lstinputlisting[language=C, firstline=37, lastline=41]{../exploit/miniclisp.c}

\subsection[Read-Loop]{Die read-eval-print-Loop}
Unsere \texttt{main}-Funktion sieht wie folgt aus:
\begin{lstlisting}[language=C]
while(1) {
  ...
  fgets(inputbuf, MAXINPUT, stdin);
  ...
  char *ptr = inputbuf;
  ...
  print_expr(eval(read(&ptr), global_env));
}
\end{lstlisting}
Wir lesen den Commandlineinput mit \texttt{fgets} in einen Buffer ein (an dieser
Stelle haben wir \emph{keinen} Bufferoverflow) und übergeben diesen dann, in
der letzten Zeile zuerst an die \texttt{read}-Funktion, die das Parsing und
Tokenizing übernimmt und die, ausgehend von erzeugten Tokens, neue 
\texttt{expr}s anlegt. \texttt{read} gibt bei jedem Aufruf eine geparste und 
neu erzeugte Expression zurück, die anschließend an \texttt{eval} übergeben
wird. \texttt{eval} gibt anschließend eine komplett ausgewertete Expression
zurück, deren Wert wir in dem Commandlineinterface ausgeben.
\par

\subsection[eval]{Die eval-Funktion}
\begin{lstlisting}[language=C]
expr *eval(expr *e, env *en)
\end{lstlisting}
Der Sinn von \texttt{eval} ist es, die als Argument übergebene Expression mit
dem gegebenen Environment-Mapping rekursiv auszuwerten. Hierbei muss je nach
Typ der Expression eine andere Auswertungstechnik verwendet werden.
Praktischerweise sind mit den Forms, die wir mit unserem Interpreter 
unterstützen, sämtliche Standardfälle von Scheme abgedeckt und es lassen sich
theoretisch beliebige Scheme-Expressions auswerten.
\pre
Die unterschiedlichen Auswertungstechniken sind folgende:
\begin{itemize}
  \item Symbolic Expressions aka Variablen:
    \begin{lstlisting}[language=C]
if (e->type == EXPRSYM) {
  expr *res = find_in_dict(e, en);
  ...
  expr *copy = create_expr(EXPRSYM);
  memcpy(copy, res, sizeof(expr));
  return copy;
}
    \end{lstlisting}
    Falls \texttt{e} eine Variable ist, suchen wir deren Wert, indem wir
    rekursiv sämtliche Environments durchgehen (angefangen bei dem lokalen bis
    hin zum globalen). Anschließend wird hier eine Kopie des Wertes
    zurückgegeben, damit der Eintrag im Environment nicht durch nachfolgende
    \texttt{eval}s verändert wird. Bestehende Werte sollten nur mit der
    Operation \texttt{set!} überschrieben werden können.
  \item Falls die Expression, die ausgewertet werden soll, vom Typ
    \texttt{EXPRLIST} ist, gibt es wiederum mehrere Alternativen. Falls der
    erste Eintrag der Liste wieder eine Symbolic Expression ist und falls
    deren \texttt{symvalue}, d.h. Name, dem einer eingebauten Expression
    (\texttt{define}, \texttt{lambda}, etc.) entspricht, müssen wir
    dementsprechenend eine spezielle Auswertungstechnik anwenden. Andernfalls
    genügt es, jede einzelne Expression in der List auszuwerten. Für letzteres
    existiert die \texttt{evalList}-Funktion. Auf die Spezialfälle wird im
    Folgenden kurz eingegangen:
    \begin{itemize}
        \item \texttt{if}: Falls wir es mit einem if-Branch zu tun haben,
          müssen wir die folgenden drei Expressions in der Liste betrachten:
\begin{lstlisting}[language=Scheme]
(if (cond) (true_expr) (false_expr))
\end{lstlisting}
          Die \emph{Condition} muss ausgewertet werden und, jenachdem ob als
          Ergebnis die vordefinierte Symbolic Expression \texttt{\#t} oder
          \texttt{\#f} zurückgegeben wird, werten wir entweder die dritte
          oder die vierte Expression aus.
        \item \texttt{begin}: Dieser Operator wertet eine nachfolgende
          Liste von Expressions aus und gibt den Wert der letzten ausgewerteten
          Expression zurück. In diesem Fall setzen wir einfach der auf die
        Symbolic Expression mit dem Namen \texttt{begin} zeigt, eine Expression
        weiter nach vorne in der Liste und werten die so neu entstandene Liste
        mit \texttt{evalList} rekursiv aus.
        \item \texttt{lambda}: Falls wir eine Expression der folgenden Form
          vorliegen haben, müssen wir daraus eine neue Lambda-Expression
          konstruieren:
\begin{lstlisting}{language=Scheme}
(lambda (arguments) (body))
\end{lstlisting}
        In diesem Fall legen wir mit \texttt{create\_expr} eine neue
        \texttt{EXPRLAMBDA} an und lassen deren \texttt{lambdavars}-Feld auf
        die \texttt{arguments} zeigen und das \texttt{lambdaexpr}-Feld auf den 
        \texttt{body}. Diese neue Lambda-Expression geben wir anschließend
        zurück.
    \end{itemize}
  \item Sollten wir es mit einer Expression vom Typ \texttt{EXPRLAMBDA} zu tun 
    haben, so müssen wir ein neues Environment anlegen, welches die lokalen
    Bindings für die Closure speichert. Und wir müssen alle Argumente, die in
    der Expressionliste \texttt{lambdavars} gespeichert sind, zusammen mit
    deren Werten, die als ausgewertete Expressionliste auf die 
    Lambda-Expression folgt, als key-value-Paare in das Environment eintragen.
    Jetzt können wir die eigentliche Prozedur, die als Pointer
    \texttt{lambdaexpr} gespeichert wurde, mit dem neuen Environment ausführen.
    Anschließend speicher wir noch das neu erzeugte Environment in dem Feld
    \texttt{lambdaenv} in der aktuellen Lambdaexpression, damit es für weitere
    geklammerte lambdas zur Verfügung steht.
\end{itemize}
