\section[Mini-C-Lisp]{Der Mini-C-Lisp-Interpreter}
Der Mini-C-Lisp-Interpreter
\footnote{Das müsste eigentlich Mini-Scheme-Interpreter heißen, aber wir
haben uns recht frühzeitig während der Entwicklungsphase auf diesen Namen
für unseren Interpreter geeinigt und auch das Git-Repository danach benannt.
}
ist in reinem C geschrieben und verwendet nur die
C-Standard-Library. Damit haben wir eine hohe Portabilität und einen sehr
einfachen Compile-Vorgang. Zusätzlich hat uns die Verwendung von C das Einbauen
von einer Sicherheitslücke erleichtert.
\par
\label{lispy}
Inspiriert hat uns der \emph{Lispy}-Interpreter von Peter Norvig
(\url{http://norvig.com/lispy.html}), der es in beeindruckenden 90 Zeilen Python-Code
fertigbringt einen funktionierenden Lisp-Interpreter zu schreiben.
\par
Eine weitere Implementierung, die wir uns angeschaut hatten, war ein 
Lispinterpreter in C++ ebenfalls in 90 LoC:
\url{http://bit.ly/1z6MagW} Letztere hat auf Norvigs \emph{Lispy} aufgebaut.

\subsection[Quellcode]{Quellcode}
Der Quellcode für den Interpreter inklusive Exploit, sowie die \LaTeX-Sources
für diese Dokumentation und die Presentation ist frei verfügbar 
über folgendes Github-Repository: 
\url{https://github.com/michaelopitz/miniclisp/}.
\par
Der eigentliche Interpreter (\texttt{miniclisp.c}) findet sich im Subdirectory
\texttt{exploit}.
\par

\subsection[Compiling]{How to Compile}
Entweder das mitgelieferte Makefile verwenden, oder compilen mit
\lstinputlisting[language=Scheme, firstline=2, lastline=2]{../exploit/Makefile}

\subsection[Funktionsweise]{Funktionsweise}
Der Mini-C-Lisp-Interpreter kann ganz normal mit \texttt{./miniclisp}
gestartet werden. Der Prozess akzeptiert dann Schemecode von stdin
und versucht diesen zu interpretieren.
\par
Das Commandline-IO ist sehr minimalistisch und kann keinen Codeinput
auswerten, der über mehrere Zeilen geht. \texttt{Enter} bzw. ein
\texttt{newline}-Character beenden den Codeinput und starten den eigentlichen
Interpreter. Können die eingegebenen Ausdrücke nicht ausgewertet werde, so
beendet der Interpreter sofort mit einer Fehlermeldung und muss dann wieder
neugestartet werden.
\par
Eingebaute Forms (speziell ausgewertete Expressions) sind:
\begin{lstlisting}[language=Scheme]
define set! lambda begin if
\end{lstlisting}
Hier haben wir den \emph{Lispy} von Peter Norvig als Referenz genommen und
dieselben implementiert, die die einfache Variante von dem \emph{Lispy}
unterstützt hat.
\par
Es sind noch einige wenige weitere manuell definierte Funktionen verfügbar:
\begin{lstlisting}[language=Scheme]
+ * < >
\end{lstlisting}
Damit können wir als \emph{Proof of Concept} einfache Integer-Arithmetik
ausführen und z.B. die \texttt{faculty}-Funktion aus der Einleitung in
Scheme implementieren. Hier wäre jedoch noch sehr viel potentieller
Erweiterungsbedarf, der für unseren eigentlichen Exploit nicht relevant ist.

\subsection[Features]{Fehlende Scheme-Features}
Unserem Interpreter fehlen neben diversen Standardfunktionen und
Datenstrukturen noch weiter wichtige Scheme-Features. Der Standard für Scheme
ist in den \emph{Revised Reports of Scheme}
\footnote{Der fünfte Report $\text{R}^5$RS ist verfügbar unter:
\url{http://www.schemers.org/Documents/Standards/R5RS/} und die sechste Version
R$^6$RS findet sich hier: \url{http://www.r6rs.org/}}
festgelegt und wird hier zum Vergleich verwendet.
\begin{itemize}
  \item Der Scheme-Standard sieht Intergerarithmetik mit beliebiger Präzision
    vor. Dieses Features wird von unserem Mini-C-Lisp nicht unterstützt, da
    wir uns auf die Standardlibrary für C beschränkt haben und somit nicht
    \emph{GMP} (\url{https://gmplib.org})) oder vergleichbares verwenden
    konnten.
  \item Optimierung von endrekursiven Funktionen wird ebenfalls vom Standard
    vorgeschrieben und ist in unserer Implementierung nicht enthalten.
  \item Scheme unterstützt keine manuelle Speicherverwaltung. Von daher
    müsste jeder Schemeinterpreter theoretisch eine Garbage-Collection
    implementieren. Der MiniCLisp-Interpreter beinhaltet eine halbfertig
    implementierte \emph{Mark-and-Sweep} Garbage-Collection. Diese kann
    im Interpreter mit dem Befehl \texttt{(gc)} aufgerufen werden. Davon
    raten wir jedoch ab, da die aktuelle Implementierung diverse wichtige
    Datenstrukutren nicht korrekt freigibt oder zerstört und beim weiteren
    Ausführen des Interpreters zu einem Crash führen wird.
\end{itemize}

