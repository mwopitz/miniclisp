\section[Mini-C-Lisp]{Der Mini-C-Lisp-Interpreter}
Der Mini-C-Lisp-Interpreter  \footnote{Das müsste eigentlich Mini-Scheme-Interpreter heißen, aber wir
  haben uns recht frühzeitig während der Entwicklungsphase auf diesen Namen
  für unseren Interpreter geeinigt und auch das Git-Repository danach benannt.
} ist in reinem C geschrieben und verwendet nur die
C-Standard-Library. Damit haben wir eine hohe Portabilität und einen sehr
einfachen Compile-Vorgang. Zusätzlich hat uns die Verwendung von C das Einbauen
von einer Sicherheitslücke erleichtert.
\par
Inspiriert hat uns der \emph{Lispy}-Interpreter von Peter Norvig
(\url{norvig.com/lispy.html}), der es in beeindruckenden 90 Zeilen Python-Code
fertigbringt einen funktionierenden Lisp-Interpreter zu schreiben.

\subsection[Quellcode]{Quellcode}
Der Quellcode für den Interpreter inklusive Exploit, sowie die \LaTeX-Sources
für diese Dokumentation und die Presentation ist frei verfügbar 
über folgendes Github-Repository: 
\url{https://github.com/michaelopitz/miniclisp/}.
\par
Der eigentliche Interpreter (\texttt{miniclisp.c}) findet sich im Subdirectory
\texttt{exploit}.
\par
Eine compilete und


\subsection[Compiling]{How to Compile}
Entweder das mitgelieferte Makefile verwenden, oder compilen mit
\lstinputlisting[language=Scheme, firstline=2, lastline=2]{../exploit/Makefile}

\subsection[Funktionsweise]{Funktionsweise}
Der Mini-C-Lisp-Interpreter kann ganu


