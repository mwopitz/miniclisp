\section[Einführung]{Einführung}
Nachdem wir ausführlich über ein Projekt für das Praktikum Binary Explotation
nachgedacht haben, ist die Entscheidung schließlich auf einen Interpreter
gefallen. Dieses Projekt erschien uns recht interessant, da beispielsweise
die JavaScript-Interpreter in Webbrowsern eine kritische Angriffsfläche bieten
und häufig das Ziel von Exploits sind.
\par
Da wir im zeitlichen Rahmen von der Projektarbeit keinen vollständigen
JavaScript-Interpreter schreiben konnten und unsere Erfahrung mit
Interpreter-Entwicklung offengestanden eher geringfügig war, haben wir uns
nach eleganten alternativen Skriptsprachen umgeschaut.
Wir haben uns schließliche darauf festgelegt, einen Scheme-Interpreter zu
schreiben, was uns verschiedene Teile der Programmierarbeit erleichtert hat:
\begin{enumerate}
  \item Hauptargument für die Verwendung von Scheme war, dass sich diese
    Skriptsprache als LISP-Dialekt durch eine sehr minimalistische Syntax
    auszeichnet.
    \lstinputlisting[language=Scheme, firstline=0]{scheme-example1.txt}
    Der Fakt dass jeder Ausdruck in Scheme geklammert ist, vereinfacht das
    Parsing und Tokenizing der Sprache enorm.
    \par
    Dazu kommt, dass Scheme-Code praktische nur aus
    sog. \emph{S-Expressions} besteht, d.h.
    jeder Ausdruck (S-Expression) ist entweder ein atomarer Ausdruck oder
    ein Operator mit einer Liste von weiteren Ausdrücken. Diese Struktur
    ermöglicht es, beim Auswerten der Ausdrücke mit nur sehr wenigen Spezialfällen auszukommen.
    Lediglich die speziell ausgewerteten Ausdrücke \texttt{define},
    \texttt{lambda}, \texttt{if}, \texttt{begin}, \texttt{quote}, \texttt{set!}
    sowie einige wenige weitere, die auf den zuvor genannten aufbauen, müssen
    im Interpreter eingebaut werden.
  \item Scheme ist nicht objektorientiert und unterstützt standardmäßig noch
    nicht einmal Funktionen. Jedoch können mit Lambda-Ausdrücken eigene 
    Prozeduren geschrieben werden und schon durch das Implementieren
    von Lambda-Ausdrücken, kann ein Scheme-Interpreter eine Turing-vollständige
    Teilsprache von Scheme auswerten. Damit war es uns möglich, einen lauffähigen
    Interpreter zu schreiben, der zur Demonstrationszwecken nur einen kleinen
    Teil der Sprache Scheme unterstützt, aber trotzdem komplexe Prozeduren
    verwenden kann.
\end{enumerate}
