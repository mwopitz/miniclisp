% Using ACM SIG Proceedings Template:
% http://www.acm.org/sigs/publications/proceedings-templates
\documentclass{sig-alternate}

\usepackage[utf8]{inputenc}
\usepackage[ngerman]{babel}

\usepackage{listings}
\usepackage{color}
\usepackage{textcomp}

\usepackage{hyperref}
\hypersetup{colorlinks,urlcolor=blue}

% Remove stupid ACM copyright notice.
\makeatletter
\def\@copyrightspace{\relax}
\makeatother


\begin{document}

\title{Binary Exploitation Projekt: Scheme Interpreter}

\numberofauthors{8}

\author{
\alignauthor
Andreas Ruhland\email{ruhland@in.tum.de}
\alignauthor
Michael Opitz\email{opitz@in.tum.de}
}

\maketitle

\section[Einführung]{Einführung}
Nachdem wir ausführlich über ein Projekt für das Praktikum Binary Explotation
nachgedacht haben, ist die Entscheidung schließlich auf einen Interpreter
gefallen. Dieses Projekt erschien uns recht interessant, da beispielsweise
die JavaScript-Interpreter in Webbrowsern eine kritische Angriffsfläche bieten
und häufig das Ziel von Exploits sind.
\par
Da wir im zeitlichen Rahmen von der Projektarbeit keinen vollständigen
JavaScript-Interpreter schreiben konnten und unsere Erfahrung mit
Interpreter-Entwicklung offengestanden eher geringfügig war, haben wir uns
nach eleganten alternativen Skriptsprachen umgeschaut.
Wir haben uns schließliche darauf festgelegt, einen Scheme-Interpreter zu
schreiben, was uns verschiedene Teile der Programmierarbeit erleichtert hat:
\begin{enumerate}
  \item Hauptargument für die Verwendung von Scheme war, dass sich diese
    Skriptsprache als LISP-Dialekt durch eine sehr minimalistische Syntax
    auszeichnet.
    \lstinputlisting[language=Scheme, firstline=0]{scheme-example1.txt}
    Der Fakt dass jeder Ausdruck in Scheme geklammert ist, vereinfacht das
    Parsing und Tokenizing der Sprache enorm.
    \par
    Dazu kommt, dass Scheme-Code praktische nur aus
    sog. \emph{S-Expressions} besteht, d.h.
    jeder Ausdruck (S-Expression) ist entweder ein atomarer Ausdruck oder
    ein Operator mit einer Liste von weiteren Ausdrücken. Diese Struktur
    ermöglicht es, beim Auswerten der Ausdrücke mit nur sehr wenigen Spezialfällen auszukommen.
    Lediglich die speziell ausgewerteten Ausdrücke \texttt{define},
    \texttt{lambda}, \texttt{if}, \texttt{begin}, \texttt{quote}, \texttt{set!}
    sowie einige wenige weitere, die auf den zuvor genannten aufbauen, müssen
    im Interpreter eingebaut werden.
  \item Scheme ist nicht objektorientiert und unterstützt standardmäßig noch
    nicht einmal Funktionen. Jedoch können mit Lambda-Ausdrücken eigene 
    Prozeduren geschrieben werden und schon durch das Implementieren
    von Lambda-Ausdrücken, kann ein Scheme-Interpreter eine Turing-vollständige
    Teilsprache von Scheme auswerten. Damit war es uns möglich, einen lauffähigen
    Interpreter zu schreiben, der zur Demonstrationszwecken nur einen kleinen
    Teil der Sprache Scheme unterstützt, aber trotzdem komplexe Prozeduren
    verwenden kann.
\end{enumerate}

\section[Mini-C-Lisp]{Der Mini-C-Lisp-Interpreter}
Der Mini-C-Lisp-Interpreter  \footnote{Das müsste eigentlich Mini-Scheme-Interpreter heißen, aber wir
  haben uns recht frühzeitig während der Entwicklungsphase auf diesen Namen
  für unseren Interpreter geeinigt und auch das Git-Repository danach benannt.
} ist in reinem C geschrieben und verwendet nur die
C-Standard-Library. Damit haben wir eine hohe Portabilität und einen sehr
einfachen Compile-Vorgang. Zusätzlich hat uns die Verwendung von C das Einbauen
von einer Sicherheitslücke erleichtert.
\par
Inspiriert hat uns der \emph{Lispy}-Interpreter von Peter Norvig
(\url{norvig.com/lispy.html}), der es in beeindruckenden 90 Zeilen Python-Code
fertigbringt einen funktionierenden Lisp-Interpreter zu schreiben.

\subsection[Quellcode]{Quellcode}
Der Quellcode für den Interpreter inklusive Exploit, sowie die \LaTeX-Sources
für diese Dokumentation und die Presentation ist frei verfügbar 
über folgendes Github-Repository: 
\url{https://github.com/michaelopitz/miniclisp/}.
\par
Der eigentliche Interpreter (\texttt{miniclisp.c}) findet sich im Subdirectory
\texttt{exploit}.
\par
Eine compilete und


\subsection[Compiling]{How to Compile}
Entweder das mitgelieferte Makefile verwenden, oder compilen mit
\lstinputlisting[language=Scheme, firstline=2, lastline=2]{../exploit/Makefile}

\subsection[Funktionsweise]{Funktionsweise}
Der Mini-C-Lisp-Interpreter kann ganu

\label{implementierung}
\section[Implementierung]{Implementierung}

Wir haben uns bei der Entwicklung von Interpreter an der Referenzimplementierung
von Peter Norvig \ref{lispy} orientiert und eine ähnliche grundliegende
Struktur verwendet. Sowohl unser MiniCLisp-Interpreter als auch Norvigs
\emph{Lispy} verwenden eine einfache \emph{Read-eval-print-Schleife}, wobei
die wichtigste und interessanteste Funktion in beiden Implementierungen die
\texttt{eval}-Funktion ist, da in dieser die Eigentliche Scheme-Logik
umgesetzt wird.

\subsection[Datenstrukturen]{Verwendete Datenstrukturen}
Die zwei wichtigsten Datenstrukturen, die unser Interpreter verwendet, sind 
\emph{Expressions} (\texttt{struct expr}) und \emph{Environments} (\texttt{struct env}).
\par
Expressions werden verwendet um die typischen \emph{S-Expressions} von
Lisp und Scheme zu implementieren und sie können von der \texttt{eval}-Funktion
ausgewertet werden. Das \texttt{struct expr} sieht wie folgt aus:
\lstinputlisting[language=C, firstline=20, lastline=35]{../exploit/miniclisp.c}
Jede \texttt{expr} hat einen von mehreren Type der in dem Feld \texttt{type}
steht. Folgende Typen von Expressions verwendet unsere Implementierung:
\begin{lstlisting}[language=C]
enum exprtype {
  EXPRPROC, EXPRSYM, EXPRINT,
  EXPRLAMBDA, EXPRLIST, EXPREMPTY
};
\end{lstlisting}
Jenachdem welchen Type eine Expression hat, arbeiten wir jeweils nur mit einem
Feld aus der \texttt{union} innerhalb der \texttt{expr}:
\begin{itemize}
    \item Leere Expressions (d.h. \texttt{()}) haben den Typ \texttt{EXPREMPTY}
      und verwenden keines der Felder aus der \texttt{union}.
    \item Integerexpressions haben den Typ \texttt{EXPRINT} und verwenden den
      \texttt{intvalue}.
    \item Symbolic Expressions, d.h. Variablen, verwenden das
      \texttt{char}-Array \texttt{symvalue} für den Variablennamen.
    \item Listen von Expressions handhaben wir, indem wir eine \texttt{expr}
      mit dem Typ \texttt{EXPRLIST} anlegen und den \texttt{listptr} auf das
      erste Element in einer einfach verketteten Liste von \texttt{expr}s
      zeigen lassen. Für diese Liste setzen wir dann die \texttt{next}-Pointer
      korrekt und können sie so traversieren.
    \item Für Lambda-Expressions (\texttt{EXPRLAMBDA}) verwenden wir das
      \texttt{struct} innerhalb der \texttt{union} und setzen die Felder
      \texttt{lambdavars} auf eine Liste von \texttt{EXPRSYM}s (die Variablen
      für die Prozedur) und \texttt{lambdaexpr} auf eine \texttt{expr}, die
      den eigentlichen Körper der Prozedur darstellt. Mit dem \texttt{env}-Pointer
      \texttt{lambdaenv} referenzieren wir das Environment, welches beim
      erstmaligen Auswerten einer Lambdaexpression angelegt wird, damit es für
      verschachtelte Lambdaexpressions zur Verfügung steht.
    \item Für Expressions vom Typ \texttt{EXPRPROC} (\emph{Procedures}) setzen wir den
      Funktionspointer \texttt{proc}, womit wir im Prinzip beliebige
      C-Funktionen für den Interpreter zur Verfügung stellen können.
      Diese Art von Expressions spielt für unseren Exploit später eine
      wichtige Rolle.
      Der Benutzer von dem MiniCLisp-Interpreter kann auf diese Weise keine
      eigenen Prozeduren definieren, sondern muss für auf Lambdas
      zurückgreifen. Wir stellen jedoch mittels \texttt{EXPRPROC}s dem 
      Benutzer funktionen \texttt{+}, \texttt{*}, \texttt{<}, \texttt{gc}, usw.
      zur Verfügung.
\end{itemize}
Das \texttt{in\_use}-Flag ist für die (nicht funktionierende) Garbage-Collection
notwendig.
\par
Die zweite wichtige Datenstruktur, die wir in unserem Interpreter verwenden,
sind Environments, die ein Mapping zwischen Variablennamen und deren Wert
darstellen. Diese werden insbesondere für \emph{Closures} benötigt, d.h. wenn
man bespielweise mehrer geklammerte Lambda-Expressions aufruft
\begin{lstlisting}[language=Scheme]
    (define f (lambda (a) (lambda (b) (+ a b))))
    ((f 1) 2)
\end{lstlisting}
Wir müssen, in diesem Fall, für die innere Lambda-Expression ein Environment
speichern, welches das Mapping \texttt{a -> 1} beinhaltet.
\par
Environments sind bei uns als simple \texttt{struct}s realisiert:
\lstinputlisting[language=C, firstline=43, lastline=47]{../exploit/miniclisp.c}
Jedes Environemnt enthält einen Pointer \texttt{outer} auf das umschließende
Environment (so dass wir bis hoch zum globalen Environment gehen können, wenn
ein Variablenname aufgelöst werden muss) und einen Pointer \texttt{list} der
auf ein simples Dictionary, in welchem die Mappings von Symbolnamen auf 
Symbolwerte gespeichert werden:
\lstinputlisting[language=C, firstline=37, lastline=41]{../exploit/miniclisp.c}

\subsection[Read-Loop]{Die read-eval-print-Loop}
Unsere \texttt{main}-Funktion sieht wie folgt aus:
\begin{lstlisting}[language=C]
while(1) {
  ...
  fgets(inputbuf, MAXINPUT, stdin);
  ...
  char *ptr = inputbuf;
  ...
  print_expr(eval(read(&ptr), global_env));
}
\end{lstlisting}
Wir lesen den Commandlineinput mit \texttt{fgets} in einen Buffer ein (an dieser
Stelle haben wir \emph{keinen} Bufferoverflow) und übergeben diesen dann, in
der letzten Zeile zuerst an die \texttt{read}-Funktion, die das Parsing und
Tokenizing übernimmt und die, ausgehend von erzeugten Tokens, neue 
\texttt{expr}s anlegt. \texttt{read} gibt bei jedem Aufruf eine geparste und 
neu erzeugte Expression zurück, die anschließend an \texttt{eval} übergeben
wird. \texttt{eval} gibt anschließend eine komplett ausgewertete Expression
zurück, deren Wert wir in dem Commandlineinterface ausgeben.
\par

\subsection[eval]{Die eval-Funktion}
\begin{lstlisting}[language=C]
expr *eval(expr *e, env *en)
\end{lstlisting}
Der Sinn von \texttt{eval} ist es, die als Argument übergebene Expression mit
dem gegebenen Environment-Mapping rekursiv auszuwerten. Hierbei muss je nach
Typ der Expression eine andere Auswertungstechnik verwendet werden.
Praktischerweise sind mit den Forms, die wir mit unserem Interpreter 
unterstützen, sämtliche Standardfälle von Scheme abgedeckt und es lassen sich
theoretisch beliebige Scheme-Expressions auswerten.
\pre
Die unterschiedlichen Auswertungstechniken sind folgende:
\begin{itemize}
  \item Symbolic Expressions aka Variablen:
    \begin{lstlisting}[language=C]
if (e->type == EXPRSYM) {
  expr *res = find_in_dict(e, en);
  ...
  expr *copy = create_expr(EXPRSYM);
  memcpy(copy, res, sizeof(expr));
  return copy;
}
    \end{lstlisting}
    Falls \texttt{e} eine Variable ist, suchen wir deren Wert, indem wir
    rekursiv sämtliche Environments durchgehen (angefangen bei dem lokalen bis
    hin zum globalen). Anschließend wird hier eine Kopie des Wertes
    zurückgegeben, damit der Eintrag im Environment nicht durch nachfolgende
    \texttt{eval}s verändert wird. Bestehende Werte sollten nur mit der
    Operation \texttt{set!} überschrieben werden können.
  \item Falls die Expression, die ausgewertet werden soll, vom Typ
    \texttt{EXPRLIST} ist, gibt es wiederum mehrere Alternativen. Falls der
    erste Eintrag der Liste wieder eine Symbolic Expression ist und falls
    deren \texttt{symvalue}, d.h. Name, dem einer eingebauten Expression
    (\texttt{define}, \texttt{lambda}, etc.) entspricht, müssen wir
    dementsprechenend eine spezielle Auswertungstechnik anwenden. Andernfalls
    genügt es, jede einzelne Expression in der List auszuwerten. Für letzteres
    existiert die \texttt{evalList}-Funktion. Auf die Spezialfälle wird im
    Folgenden kurz eingegangen:
    \begin{itemize}
        \item \texttt{if}: Falls wir es mit einem if-Branch zu tun haben,
          müssen wir die folgenden drei Expressions in der Liste betrachten:
\begin{lstlisting}[language=Scheme]
(if (cond) (true_expr) (false_expr))
\end{lstlisting}
          Die \emph{Condition} muss ausgewertet werden und, jenachdem ob als
          Ergebnis die vordefinierte Symbolic Expression \texttt{\#t} oder
          \texttt{\#f} zurückgegeben wird, werten wir entweder die dritte
          oder die vierte Expression aus.
        \item \texttt{begin}: Dieser Operator wertet eine nachfolgende
          Liste von Expressions aus und gibt den Wert der letzten ausgewerteten
          Expression zurück. In diesem Fall setzen wir einfach der auf die
        Symbolic Expression mit dem Namen \texttt{begin} zeigt, eine Expression
        weiter nach vorne in der Liste und werten die so neu entstandene Liste
        mit \texttt{evalList} rekursiv aus.
        \item \texttt{lambda}: Falls wir eine Expression der folgenden Form
          vorliegen haben, müssen wir daraus eine neue Lambda-Expression
          konstruieren:
\begin{lstlisting}{language=Scheme}
(lambda (arguments) (body))
\end{lstlisting}
        In diesem Fall legen wir mit \texttt{create\_expr} eine neue
        \texttt{EXPRLAMBDA} an und lassen deren \texttt{lambdavars}-Feld auf
        die \texttt{arguments} zeigen und das \texttt{lambdaexpr}-Feld auf den 
        \texttt{body}. Diese neue Lambda-Expression geben wir anschließend
        zurück.
    \end{itemize}
  \item Sollten wir es mit einer Expression vom Typ \texttt{EXPRLAMBDA} zu tun 
    haben, so müssen wir ein neues Environment anlegen, welches die lokalen
    Bindings für die Closure speichert. Und wir müssen alle Argumente, die in
    der Expressionliste \texttt{lambdavars} gespeichert sind, zusammen mit
    deren Werten, die als ausgewertete Expressionliste auf die 
    Lambda-Expression folgt, als key-value-Paare in das Environment eintragen.
    Jetzt können wir die eigentliche Prozedur, die als Pointer
    \texttt{lambdaexpr} gespeichert wurde, mit dem neuen Environment ausführen.
    Anschließend speicher wir noch das neu erzeugte Environment in dem Feld
    \texttt{lambdaenv} in der aktuellen Lambdaexpression, damit es für weitere
    geklammerte lambdas zur Verfügung steht.
\end{itemize}

\section{Exploit}
In dem Interpreter befindet sich ein sogennanter Off-By-One Fehler. Fehler dieser Art können sich sehr schnell in ein Programm einschleichen. Ein typisches Beispiel wäre dass der Programmierer die Länge eines Array und den Index des letzten Elements gleichsetzt obwohl die Nummerierung bei C mit 0 beginnt. Off-By-One Fehler sind nur unter sehr bestimmten Vorraussetzungen exploit-bar und sind dadurch von Programm zu Programm sehr unterschiedlich. Deshalb können kleine Änderungen an dem Programm schnell dazu führen, dass es nicht mehr exploitbar oder der Exploit komplett anders zu gestalten ist.
\subsection{Aktivierte Sicherheitsmechanismen}
Das Exploit kann mit einigen üblichen Sicherheitsmechanismen ausgeführt werden. Zur Entwicklung wurde ein Ubuntu 10.10 und ein Arch Linux\footnote{Neuste Version vom 1.6.2014 z.B. kernel: 3.15.2, gcc-Version 4.9.0, glibc 2.19} benutzt. Dabei war ASLR standardmäßig aktiviert und der Stack nicht ausführbar (ebenso Standard im gcc). Zusätzlich ist der GCC StackGuard aktiviert was jedoch keine Rolle spielt für das Exploit. Nicht aktiviert sind Sicherheitsmechanismen wie "position independent executable"(-fPIE).
\\
Diese Sicherheitskonfiguration spiegelt auch aktuelle Systeme sehr gut wieder und ist auch häufig anzutreffen. Um die Gefahr solcher Lücken zu demonstrieren wurde die binary mit einem speziellen Benutzer und mit dem Setuid-Flag platziert. Das Setuid-Flag wäre in der aktuellen Konfiguration nicht unbedingt notwendig aber ein Szenario in dem ein Benutzer eine Scheme Datei aus unsicherer Quelle mit eigenem Bneutzer interpretiert wäre vergleichbar. Die Datei wurde auf dem hacky2 Rechner des Binary Exploitation Praktikums der TU-München installiert und als Eigentümer wurde der Benutzer "r00t" verwendet.

\subsection{Der Off-By-One Fehler}
\label{luecke}
Der Off-By-One Fehler geschieht in der der read Funktion.\\
 Jedes Token das eingelesen wird hat eine maximale Länge (\texttt{MAXTOKENLEN}). Das bedeutet eine neu angelegte Variable z.B. mit define darf nur begrentzt lang sein.
Das String-Array das ein Symbol speichern soll hat genau diese Länge: \texttt{char symvalue[MAXTOKENLEN]}. Direkt in der read Funktion wird bei zu großen Tokens ein Fehler ausgegeben, falls der Check if \texttt{tokenlen > MAXTOKENLEN} fehlschlägt. 
Hier ist bereits der Fehler zu erkennen da dass Array einen C-String der Länge \texttt{tokenlen} nicht speichern kann, da von den restlichen Programmteilen ein normaler C-String mit Null Terminierung erwartet wird. 
Die neue Expression wird dann schließlich in der Funktion \texttt{create\_exprsym(const char *s)} erstellt. Hier wird zuerst eine leere Expression erzeugt und dann mit Hilfe von 
\begin{lstlisting} 
strncat(new->symvalue, s, MAXTOKENLEN);
\end{lstlisting} mit dem String gefüllt. Ist nun der String s genau \texttt{MAXTOKENLEN} lang fügt \texttt{strncat} nach dieser Länge ein 0 Byte an, dass sich dann nicht mehr innerhalb des Arrays befindet. \\
Wie bereits in \ref{datenstrukturen} beschrieben befindet sich nach der symvalue direkt der Typ der Expression: \texttt{enum exprtype type}. Dieser wird jetzt mit dem 0 Byte von strncat überschrieben und so wird aus einer Expression des Types: \texttt{EXPRSYM} eine Expression mit dem Typen: \texttt{EXPRPROC}.\\
Wird diese Expression nun interpretiert liefert sie anstatt dem Variablenwert des Symbols eine Prozedur die anschließend ausgeführt wird. Da unions verwendet wurden werden nun die ersten 4 bytes des strings als Adresse zu einer Prozedur interpretiert.
Beispielsweise liefert folgende Eingabe
\begin{lstlisting}
> (define test abcdeeeeeeffffffffff
		gggggggggghh)
> test
 PROC: 0x64636261 
\end{lstlisting}
Dies bedeutet, dass unter dem Symbol test ab jetzt eine Prozedur vergleichbar mit '+' gespeichert ist. 
Genau wie anderen Prozeduren können der Prozedur Parameter übergeben werden. 
Dies sind mehrere Expressions die mit Hilfe eines next Pointers verbunden sind. Der Prozedur werden nur aufgelöste Werte übergeben. In unserem Fall ist der einzige Basis Typ ein 64 bit langer \texttt{long long int}. \\
Dies bedeutet dass wir der Prozedur nur \texttt{long long int} übergeben können. 
Ein Zeiger auf den ersten Wert wird direkt der Prozedur übergeben und kann somit auch für C Funktionen genutzt werden die nicht  wissen was eine Expression ist. 
Nehmen wir an dass wir eine Prozedur mit der Adresse der Linux system libc Funktion definiert haben, dann könnte man mit 
\begin{lstlisting}
(define systemproc 0x68732f6e69622f)
\end{lstlisting}
die Funktion \texttt{system} aufrufen mit einem Zeiger auf den \texttt{long long int} der als \texttt{char *} interpretiert \texttt{'/bin/sh'} wird. Dadurch würde sich eine interaktive Shell öffnen.
\subsection{Bestandteile des Exploits}
Das Exploit besteht im Grunde aus 3 Teilen die gegenseitig von einander benutzt werden.
\begin{enumerate}
\item Die Bash Datei exploit.sh beinhaltet den Hauptteil des Exploits. Da es für das Exploit notwendig ist mit dem Interpreter dynamisch zu interagieren werden temporär noch einige Dateien durch dieses Bash-Skript erzeugt. 
Alternativ wäre das Paket \texttt{expect}\footnote{\url{http://expect.sourceforge.net/}} zu verwendet, was die Interaktion erleichtert jedoch auf dem Zielsystem nicht verfügbar war. 
Um die Ausgabe in die Eingabe mit einzubeziehen wird mit Hilfe einer Redirection die Ausgabe in eine Datei umgeleitet. Diese Datei wird dann während der Eingabe mit einbezogen werden. 
In Bash kann dies z.B. so realisiert werden 
\begin{lstlisting}
(	echo "a"; 
	sleep 1; 
	verarbeiteAusgabe; 
	echo verarbeiteteAusgabe
	) | ./interpreter > output.txt
\end{lstlisting}
\item Das kleine C-Programm hexto32binary nimmt 4 Parameter. Der erste Eingabe Parameter ist eine Adresse, danach das Offset was von der Adresse abgezogen werden soll. Als drittes wird wieder eine Adresse übergeben und anschließend wieder ein Offset.
Die Übergabe passiert einfach anhand eines Strings der als hex oder int eine Zahl repräsentiert. Das C-Programm berechnet die neuen Adressen mithilfe der Offsets und gibt anschließend einen genau 32 byte langen String aus.
 Die ersten 4 byte des Strings sind die ASCII Zeichen die die erste Adresse im little Endian format repräsentiert. 
Die Bytes 10-14 sind wiederum die neu berechnete 2 Adresse. Die chars 0x0, '(', ')', ' ' werden durch Fülbuchstaben ersetzt. Die restlichen Bytes werden aufgefüllt mit 'a's.
\item Das Bash-Skript s.sh beinhaltet die Shell commands die die erreichte Shell ausführen soll. Dort werden die Streams angepasst und dann eine interaktive Shell mit \texttt{'/bin/sh'} gestartet.
\end{enumerate}
\subsection{Vorgehen}
Das Exploit lässt sich in mehrer Schritte unterteilen, die im Folgenden näher beleuchtet werden sollen. Anfangs wird durch aussnutzen der Lücke beschriben in \ref{luecke} eine Prozedur \texttt{printfproc} definierti, die über die GOT die libc printf Funktion aufruft. Nun können wir printf direkt aus lisp heraus aufrufen und einen maximal 8 byte (\texttt{sizeof(long long int}) langen String übergeben.
 Zuerst rufen wir printf mit \texttt{"\%2\$p\textbackslash n"} und \texttt{"\%9\$p\textbackslash n"} auf, dies liefert eine Adresse aus der libc und eine Adresse aus dem heap, die beide randomisiert sind. 
Da die Offsets innerhalb der libc konstant sind kann durch den ersten Wert die Adresse von system berechnet werden. Da auch malloc deterministisch arbeitet kann aus der zweiten Adresse die Adresse einer bestimmten Expression im Heap berechnet werden.
\begin{lstlisting}
> (define printfproc \x60\x86\x04\x08aaa
	aaabbbbbbbbbbccccccccccdd)
> (printfproc 0xa70243225)
> 0x667440
> (printfproc 0xa70243925)
> 0x8417838
\end{lstlisting}
Wie hier auffält ist die libc Library in einen sehr niedrigen Speicherbereich gemapped wodurch die Adresse von system auch mit  \texttt{0x00} beginnt. Dadurch wird es notwendig das 0 Byte nachträglich noch in der Adresse zu setzen da innerhalb keines Symbols EOF vorkommen darf, was die Verabreitung beenden würde.
Mit Hilfe dieser Adressen werden nun die Adressen von system (hier: 0x168680) und der systemproc Expression (hier: 0x0804e9fb) die erst noch angelegt wird ermittlet.
\begin{lstlisting}
> (define systemproc \x80\x86\x16raaaaaa
	\xfb\xe9\x04\x08aaaaaaaaaaaaaaaa)
\end{lstlisting}
Das "r" muss nun noch durch eine 0 ersetzt werden. 
Dazu benutzen wir die Adresse die an Position 10 des Strings steht. 
Diese Adresse zeigt genau auf das "r" weiter vorne. 
Nun rufen wir noch einmal \texttt{printfproc} auf mit dem Format-String "\texttt{\%103\$n\textbackslash n}". Mit dem Offset von 103 wir genau die Adresse 0x0804e9fb von getroffen. Da noch keine Zeichen innerhalb des Format-Strings ausgegeben wurden wird 0 an die Stelle geschriebeni (32 bit also werden auch die drei auf das 'r' flogenden 'a's mit Nullen überschrieben.
\\
Nun rufen wir \texttt{systemproc} mit './s.sh' auf um den Inhalt der './s.sh' Datei auszuführen.
\begin{lstlisting}
> (systemproc 0x68732e732f2e)
\end{lstlisting}
In der s.sh Datei werden einige Dateiströme wieder auf stdin und stdout gesetzt und dann eine interaktive Shell geöffnet.
\subsection{Ergebnis}
Auch ein einfacher Off-By-One Fehler kann zu schwerwiegenden Folgen führen und der Compiler kann wenig dagegen unternehmen. 
Einzige Vorraussetzung war, dass die printf funktion in der plt verfügbar ist und keine Zeichen wie 0x00,'(',')',' ' enthält. Andere Konflikte wie z.B. 0x00 in der system Adresse können umgangen werden. \\
Generell ist bei Funktions-Pointern immer ein gewisses Maß an Vorsicht zu benutzen.





\end{document}
